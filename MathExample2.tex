%-%-%-%-%-%-%-%-%-%-%-%-%-%-%-%-%-%-%-%-%-%-%-%-%-%-%-%-%-%-%-%-%-%-%-%-%-%-%
%This is a blank document for homework assignments.

%Some preliminaries:  Anything after a '%' is a comment - it isn't read by 
%the compiler.  

%You are welcome to skip down to lines 38-44 to put in some information, and 
%then to line 57 to start writing, but the preamble contains all the 
%formatting that makes it look nice, if you're interested in how that works.

%Packages are just collections of commands to do different things.  For
%almost anything you might want to do, there's a package that will do it.
%-%-%-%-%-%-%-%-%-%-%-%-%-%-%-%-%-%-%-%-%-%-%-%-%-%-%-%-%-%-%-%-%-%-%-%-%-%-%


\documentclass[12pt]{article}  
%The article class is a very basic type of document for writing
%We will customize it to do what we want.

\usepackage[margin=1in]{geometry}  %Adjust margins, formatting

\usepackage{amsmath}  
\usepackage{amssymb}  
\usepackage{amsfonts}  
%These packages add commands for useful symbols and fonts and things like that.
%Most of the time, these are all you need.

\usepackage{textcomp, gensymb}  %Gives more symbols, like /degree

\usepackage{amsthm}

\usepackage{fancyhdr}  %Header and Footer formatting
\pagestyle{fancy}  
\renewcommand{\headrulewidth}{0.4pt}
\renewcommand{\footrulewidth}{0.4pt}
\setlength{\headheight}{18pt}

%Header and Footer Information
\lhead{\large{\bf Sophia Pushkarski}}  %Replace with your name
\chead{}
\rhead{\textsc{My example TeX document}}  %Replace "Title" with the name of the assignment
\lfoot{\today}  %You can let it put in today's date or put one in yourself
\cfoot{}
\rfoot{\thepage\ of \ref{NumPages}}  %Counts the pages.

\makeatletter        %This provides a total page count as \ref{NumPages}                 
\AtEndDocument{\immediate\write\@auxout{\string\newlabel{NumPages}{{\thepage}}}}
\makeatother

\usepackage{amsthm}  %This will create the Problem environment
\theoremstyle{definition}
\newtheorem{problem}{Problem}
\renewcommand*{\proofname}{Solution}

\usepackage{graphicx}



\begin{document}

\begin{problem}
Re-state the problem here.

\begin{proof}
Type your solution here.
\end{proof}
\end{problem}

%-%-%-%-%-%-%-%-%-%-%-%-%-%-%-%-%-%-%-%-%-%-%-%-%-%-%-%-%-%-%-%-%-%-%-%-%-%-%
%This is a blank document for homework assignments.

%Some preliminaries:  Anything after a '%' is a comment - it isn't read by 
%the compiler.  

%You are welcome to skip down to lines 38-44 to put in some information, and 
%then to line 57 to start writing, but the preamble contains all the 
%formatting that makes it look nice, if you're interested in how that works.

%Packages are just collections of commands to do different things.  For
%almost anything you might want to do, there's a package that will do it.
%-%-%-%-%-%-%-%-%-%-%-%-%-%-%-%-%-%-%-%-%-%-%-%-%-%-%-%-%-%-%-%-%-%-%-%-%-%-%


\documentclass[12pt]{article}  
%The article class is a very basic type of document for writing
%We will customize it to do what we want.

\usepackage[margin=1in]{geometry}  %Adjust margins, formatting

\usepackage{amsmath}  
\usepackage{amssymb}  
\usepackage{amsfonts}  
%These packages add commands for useful symbols and fonts and things like that.
%Most of the time, these are all you need.

\usepackage{textcomp, gensymb}  %Gives more symbols, like /degree

\usepackage{amsthm}

\usepackage{fancyhdr}  %Header and Footer formatting
\pagestyle{fancy}  
\renewcommand{\headrulewidth}{0.4pt}
\renewcommand{\footrulewidth}{0.4pt}
\setlength{\headheight}{18pt}

%Header and Footer Information
\lhead{\large{\bf Name}}  %Replace with your name
\chead{}
\rhead{\textsc{Title}}  %Replace "Title" with the name of the assignment
\lfoot{\today}  %You can let it put in today's date or put one in yourself
\cfoot{}
\rfoot{\thepage\ of \ref{NumPages}}  %Counts the pages.

\makeatletter        %This provides a total page count as \ref{NumPages}                 
\AtEndDocument{\immediate\write\@auxout{\string\newlabel{NumPages}{{\thepage}}}}
\makeatother

\usepackage{amsthm}  %This will create the Problem environment
\theoremstyle{definition}
\newtheorem{problem}{Problem}
\renewcommand*{\proofname}{Solution}



\begin{document}

\begin{problem}
Re-state the problem here.

\begin{proof}
Type your solution here.
\end{proof}
\end{problem}

\end{document}